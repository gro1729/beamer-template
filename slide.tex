\documentclass[14pt,t]{beamer}
\usetheme{clean}


% Packages
%\usepackage{luatexja}  % for Japanese
\usepackage{latexsym}
\usepackage{amssymb,amsmath}
\usepackage{graphicx}
\usepackage{hyperref}
%\usepackage{algorithm}  % must be behind hyperref


% Path
%\graphicspath{{image/}}


% Layout Settings
%\setlength{\voffset}{-0.5in}
%\addtolength{\textheight}{1.0in}
%\setlength{\hoffset}{-0.25in}
%\addtolength{\textwidth}{0.5in}


% Floats
%\setcounter{topnumber}{2}  % max number of floats at top of a page
%\setcounter{bottomnumber}{1}  % max number of floats at bottom of a page
%\setcounter{totalnumber}{3}  % max number of floats on a page
%\setcounter{dbltopnumber}{2}
%\renewcommand{\topfraction}{0.7}  % max fraction of page for floats at top; the bigger the more floats can be put at top of a page
%\renewcommand{\bottomfraction}{0.3}  % max fraction of page for floats at bottom; the bigger the more floats can be put at bottom of a page
%\renewcommand{\textfraction}{0.2}  % min fraction of page for text; smaller value allow LaTeX to put small amount of text on a page
%\renewcommand{\floatpagefraction}{0.5}  % min fraction of float page that should have floats; bigger value discourages LaTeX from putting a small float on a float page
%\renewcommand\dbltopfraction{.7}
%\renewcommand\dblfloatpagefraction{.5}


% Bibliography Style
%\bibliographystyle{plain}


% Mathmatical Functions
%\DeclareMathOperator{\sin}{sin}
%\DeclareMathOperator{\argmax}{argmax}


% Other Commands
\definecolor{todo-color}{rgb}{1.0,0.0,0.0}
\newcommand{\todo}[1]{{\textcolor{todo-color}{#1}}}
\newcommand{\rarrow}{$\Rightarrow$}
\newcommand{\mathbs}[1]{{\boldsymbol{#1}}}
%\newcommand{\ud}{{\,\mathrm{d}}}
\newcommand{\diff}[2]{\frac{d#1}{d#2}}
\newcommand{\pdiff}[2]{\frac{\partial#1}{\partial#2}}
%\newenvironment{fminipage}%
%{\begin{Sbox}\begin{minipage}}%
%    {\end{minipage}\end{Sbox}\fbox{\TheSbox}}


% Document Properties
\title{Hyper Text Markup Language}
\subtitle{Web Technology I}
\author{Yuta Taniguchi}
\institute{@yuttieyuttie}
\date{2013-10-29}


\begin{document}


\frame{\titlepage}
\setcounter{framenumber}{0}


\begin{frame}{Basic}
  \begin{itemize}
  \item AAA
    \begin{itemize}
    \item 111
    \item 222
    \end{itemize}
  \item BBB
    \begin{itemize}
    \item 111
    \item 222
    \end{itemize}
  \end{itemize}
  \begin{enumerate}
  \item Alpha
    \begin{enumerate}
    \item I
    \item II
    \item III
    \end{enumerate}
  \item Beta
  \item Gamma
  \end{enumerate}
  \begin{description}
  \item[Abc] basic
  \item[Xyz] extra
  \end{description}
\end{frame}


\begin{frame}[fragile]{HTML}
  \begin{columns}
    \begin{column}{0.4\textwidth}
      \begin{itemize}
        \item A \emph{markup} language
        \item Not related to complex expressions such as
          $e^{ix} = \cos x + i \sin x$ or
          $\Gamma(z) = \int_0^\infty t^{z-1} e^{-t} dt$
      \end{itemize}
    \end{column}
    \begin{column}{0.6\textwidth}
      \begin{html*}{gobble=8}
        <!DOCTYPE html>
        <html>
          <head>
            <meta charset="UTF-8" />
            <title>Title</title>
          </head>
          <body>
            Emphasize
            <strong id="target">it</strong>!
          </body>
        </html>
      \end{html*}
    \end{column}
  \end{columns}
\end{frame}


\begin{frame}{Rich Text}
  \begin{itemize}
  \item We can use \textit{italic} and \textbf{bold} styles
  \item Furthermore, \textit{\textbf{bold italic}} and \textsc{Small Capital} styles
  \item There is also Beamer's \alert{alerted text}.
  \end{itemize}
\end{frame}


\begin{frame}[fragile]{Blocks}
  \begin{block}{Generic Block}
    Generally, we can use this one.
  \end{block}

  \begin{alertblock}{Alert Block}
    You are alerted!
  \end{alertblock}

  \begin{exampleblock}{Example Block}
    Here is an example of the \verb|exampleblock|.
  \end{exampleblock}
\end{frame}


\end{document}
